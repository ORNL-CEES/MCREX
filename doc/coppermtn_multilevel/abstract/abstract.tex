\documentclass[letterpaper,12pt]{article}
\usepackage[top=1.0in,bottom=1.0in,left=1.0in,right=1.0in]{geometry}
\usepackage{verbatim}
\usepackage{amssymb}
\usepackage{amsmath}
\usepackage{amsthm}
\usepackage{graphicx}
\usepackage{tmadd,tmath}
\usepackage{longtable}
\usepackage{amsfonts}
\usepackage{amsmath}
\usepackage{tmath,tmadd}
\usepackage{algpseudocode}
\usepackage{algorithm}
\usepackage[mathcal]{euscript} 
\usepackage[usenames]{color}
\usepackage[
naturalnames = true, 
colorlinks = true, 
linkcolor = black,
anchorcolor = black,
citecolor = black,
menucolor = black,
urlcolor = blue
]{hyperref}

%%---------------------------------------------------------------------------%%
\author{Stuart R. Slattery, Thomas M. Evans, Steven P. Hamilton
\\ \href{mailto:slatterysr@ornl.gov}{\texttt{slatterysr@ornl.gov}}
}

\date{\today} \title{Multilevel Monte Carlo Solvers for Linear Systems}
\begin{document}
\maketitle

%%---------------------------------------------------------------------------%%
\abstract

Linear solvers based on Monte Carlo techniques are currently being
studied as potentially resilient alternatives to conventional
iterative methods. These solvers produce stochastic estimates of the
solutions to linear systems by estimating the terms of the Neumann
series formed by the Richardson iteration; thus providing a stochastic
realization of a relaxation method. The random walks used to estimate
the solution are a variant of Markov Chain Monte Carlo (MCMC) methods
and recent work has aimed to adapt various techniques in MCMC to
linear solver applications.

In this work we consider leveraging recent developments in multilevel
MCMC where the space over which the Markov chains are generated is
represented by a sequence of coarse and fine grids. The estimates on
the grids are then combined via superposition to achieve results with
lower statistical uncertainty than those on a single grid. When
applied to Monte Carlo linear solvers the multilevel technique
resembles the multigrid method where the Monte Carlo scheme acts as a
stochastic smoother. We will present both the Monte Carlo method for
linear systems as well as the multilevel method. We will then compare
the performance of the multilevel method against that of the
traditional method and discuss applicability to various problems.

\end{document}


