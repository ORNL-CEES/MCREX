%%---------------------------------------------------------------------------%%
%% white.tex
%%
%% White paper for MCREX proposal
%%---------------------------------------------------------------------------%%

\documentclass[10pt,letterpaper,english]{article}
\usepackage[top=1in, bottom=1in, left=1in, right=1in]{geometry}
\usepackage{fancyhdr}
\usepackage{amsmath}
\usepackage{tmath}

%%---------------------------------------------------------------------------%%

\newcommand{\vA}{\ensuremath{\ve{A}}}
\newcommand{\vb}{\ensuremath{\ve{b}}}
\newcommand{\vx}{\ensuremath{\ve{x}}}
\newcommand{\vr}{\ensuremath{\ve{r}}}
\newcommand{\vI}{\ensuremath{\ve{I}}}
\newcommand{\vH}{\ensuremath{\ve{H}}}
\newcommand{\vP}{\ensuremath{\ve{P}}}

%%---------------------------------------------------------------------------%%

\title{MCREX: Using Monte Carlo algorithms to achieve resiliency and
  performance at scale for linear and non-linear solver applications}
\author{Thomas Evans\thanks{PI}, Steven Hamilton, Wayne Joubert\\
  Oak Ridge National Laboratory
  \and Michele Benzi\\Emory University}
\date{July 10, 2012}

\begin{document}

\pagestyle{fancy}
\lhead{MCREX}
\rhead{T.M. Evans}

\maketitle

%%---------------------------------------------------------------------------%%

\section*{Proposal Objective}

The DOE LAB 12-742 FOA, {\em Resilient Extreme-Scale Solvers}, is looking for
research proposals in numerical solver algorithms that address three
components:
\begin{enumerate}
\item Advances in solvers
\item Fault tolerance and resilience at the algorithmic level
\item Performance of proposed algorithms
\end{enumerate} 
To address these topical areas we are proposing to investigate new stochastic
methods for solving linear systems or Monte Carlo Resilient, Exascale (MCREX)
solvers.  The family of methods that we have developed build on the sequential
Monte Carlo work of Halton \cite {halton_1962, halton_1994}. The methods that
we have developed use Monte Carlo to accelerate a fixed-point iteration;
therefore, we have called them Monte Carlo Synthetic Acceleration
(MCSA). Preliminary work \cite{rmc,mc} using MCSA has demonstrated that they
are at least as efficient as Jacobi-preconditioned Conjugate Gradient (PCG) on
sparse, SPD systems.

MCSA fits ideally into this call because, as described in the next section, we
can show how these algorithms can be used to address all three required
aspects of the FOA.  Furthermore, we will demonstrate how this can be achieved
by a small focused team.  We are seeking an award in the \$500K range, which
should be sufficient to perform algorithmic enhancements, demonstrate
resiliency, and do performance analysis.

%%---------------------------------------------------------------------------%%

\section*{Research Plan}

The Fixed-Point Monte Carlo Synthetic-Acceleration (MCSA) method is defined,
for the linear system, $\vA\vx = \vb$, as follows:
\begin{subequations}
  \begin{gather}
    \vx^{l+1/2} = (\vI - \vA)\vx^l + \vb\:, \\
    %%
    \vr^{l+1/2} = \vb - \vA\vx^{l+1/2}\:,\\
    %%
    \hat{\vA}\delta\vx^{l+1/2} = \vr^{l+1/2}\:,
    \label{eq:MCSA-residual_solve}\\ 
    %%
    \vx^{l+1}=\vx^{l+1/2}+\delta\vx^{l+1/2}\:.
  \end{gather}
  \label{eq:general-MCSA}
\end{subequations}
We use Monte Carlo to perform random walks through the elements of the matrix
to solve Eq.~(\ref{eq:MCSA-residual_solve}). The hat on $\vA$ indicates that
the Monte Carlo solution only approximately inverts this operator.  Thus, we
have defined a scheme in which the initial estimate of $\vx$ in each iteration
is updated using a single fixed-point iteration.  The residual is calculated
and is used as a source to estimate the error, $\delta\vx^{l+1/2}$.

As described in Refs.~\cite{rmc} and \cite{mc}, the power of this algorithm
derives from the fact that random walks are only performed where the residual
is large.  The components of the system that are converged do not require
further work.  Furthermore, as opposed to PCG, there are no symmetry
requirements for MCSA; thus, it could be an excellent alternative to GMRES,
especially for consistently-treated non-linear problems that are solved using
Newton's method.  Also, this algorithm naturally addresses resiliency because
random walk histories that are lost due to processor or system faults can be
neglected during any iteration.

The proposed research plan for this project is broken down to address each
component of the FOA:
\begin{enumerate}
\item Advances in solvers
  \begin{itemize}
  \item Investigate efficiency of MCSA on non-symmetric matrix systems.
  \item Investigate the efficiency of MCSA on non-linear solvers using
    automatic differentiation of the Jacobian matrix elements.  Numerical
    automatic differentiation will be provided by Trilinos.
  \item Investigate the extension of the method to non-stationary (Krylov
    sub-space) iteration schemes.
  \end{itemize}
\item Fault tolerance and resilience at the algorithmic level
  \begin{itemize}
  \item Incorporate MPI error signals into the algorithm to address cases of
    core failure.
  \end{itemize}
\item Performance of proposed algorithms
  \begin{itemize}
  \item Leverage work from CASL on the Multi-Set, Overlapping Domain (MSOD)
    Monte Carlo parallel decomposition method to test its efficiency for MCSA.
  \item Develop heuristic models to project its performance on projected
    heterogeneous, exa-scale computing platforms.
  \end{itemize}
\end{enumerate}
The model equations that we will utilize in all of these studies will be the
linear heat transfer equation and the non-linear, incompressible,
Navier-Stokes equations.

%%---------------------------------------------------------------------------%%

\section*{Summary}

We have proposed a highly focused, innovative project that naturally addresses
all three components of the FOA: novel algorithmic advances, resiliency, and
performance modeling.  The proposed team is 
\begin{center}
  \begin{tabular}{llr}\hline\hline
    \multicolumn{1}{c}{Team Member} & 
    \multicolumn{1}{c}{Role}        & 
    \multicolumn{1}{c}{Budget (K)} \\\hline
    %%
    Tom Evans       & PI                & 100  \\
    Steven Hamilton & Solver Algorithms & 175  \\
    Wayne Joubert   & Performance       & 100  \\
    Michele Benzi   & Solver Algorithms & 75   \\
    Other           & Resiliency        & 100  \\ \hline
    {\bf Total}     &                   & 550  \\
    \hline\hline
  \end{tabular}
\end{center}
The strength of this proposal is its focus on a single, innovative solver
technology that addresses aspects of all three pieces of the FOA.  Therefore,
we feel that a local, small team is both justified and appropriate.

%%---------------------------------------------------------------------------%%

\begin{thebibliography}{300}

\bibitem{halton_1962} 
  J.H. Halton. Sequential Monte Carlo, \textit{
    Proc. Camb. Phil. Soc.}, {\bf 58}, pp. 57-78, 1962.

\bibitem{halton_1994} 
  J.H. Halton. Sequential Monte Carlo Techniques for the
  Solution of Linear Systems, \textit{J. Sci. Comp.}, {\bf 9}(2), pp. 213-257,
  1994.
  
\bibitem{rmc}
  T.M. Evans, T.J. Urbatsch, H.Lichtenstein, and J.E. Morel. A
  residual Monte Carlo Method for Discrete Thermal Radiative
  Diffusion. \textit{J. Comp. Phys.}, {\bf 189}(2), 539--556, 2003.

\bibitem{mc} 
  T.M. Evans and S.W. Mosher.  A Monte Carlo Synthetic
  Acceleration Method for the Non-Linear, Time-Dependent Diffusion
  Equation.  \textit{International Conference on Mathematics,
    Computational Methods and Reactor Physics}, Saratoga Springs, NY,
  ISBN: 978-0-89448-069-0, American Nuclear Society, LaGrange Park,
  IL, 2009.

\end{thebibliography}

\end{document}

%%---------------------------------------------------------------------------%%
%% end of white.tex
%%---------------------------------------------------------------------------%%


