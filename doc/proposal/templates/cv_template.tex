%%---------------------------------------------------------------------------%%
%% CV For DOE Proposals
%%---------------------------------------------------------------------------%%

\documentclass[11pt]{article}

%%---------------------------------------------------------------------------%%

\pagestyle{empty}

\setlength{\textheight}{9.5in}
\setlength{\textwidth}{6.5in}
\setlength{\footskip}{0.5in}
\setlength{\headheight}{0.0in}
\setlength{\topmargin}{0.0in}
\setlength{\headsep}{0.0in}

\setlength{\oddsidemargin}{0in}
\setlength{\evensidemargin}{0in}

\setlength{\parindent}{0in}

%%---------------------------------------------------------------------------%%

\begin{document}

%%---------------------------------------------------------------------------%%

{\bf Thomas M. Evans}\\
Radiation Transport Group, Reactor and Nuclear Systems Division, Oak Ridge
National Laboratory, P.O. Box 2008, Bldg. 5700, Oak Ridge, TN 37831-6170,
$<$evanstm@ornl.gov$>$\\

%%---------------------------------------------------------------------------%%

{\bf PROFESSIONAL EDUCATION}
\begin{tabbing}
  xxxxxxxxxxxxxxxxxxxxxxxxxxxxxx \= xxxxxxxxxxxxxxxxxxxxx \= xxxxxxxxx \kill
  Haverford College              \> Physics and Astronomy \> B.S. 1992 \\
  Georgia Insitute of Technology \> Health Physcs         \> M.S. 1994 \\
  Georgia Insitute of Technology \> Nuclear Engineering   \> PhD  1997 
\end{tabbing}

%%---------------------------------------------------------------------------%%

{\bf RESEARCH AND PROFESSIONAL EXPERIENCE}\\

\underline{Radiation Transport Group, Oak Ridge National Laboratory}\hfill
2007--present\\
\textit{Senior R\&D Staff}\\
Develop computational methods, algorithms, and parallel application codes in
the Reactor and Nuclear Systems Division for ORNL customers.  Develop and
submit proposals for scientific funding.\\

\underline{Radiation Transport Group, Los Alamos National Laboratory}\hfill
1997--2007\\
\textit{Project Leader (2003--2007)}; \textit{Techical Staff Member
  (1997--2003)};
\textit{Postdoctory Fellow (1997--1997)}\\
Project Leader of the Marmot and Jayenne projects. As a staff member,
contributed to radiation transport and radiative
transfer methods development in CCS-4 Transport Methods Group. \\

%%---------------------------------------------------------------------------%%

{\bf RELEVANT PUBLICATIONS}
\begin{enumerate}
  \setlength{\itemsep}{0in}
\item G.G. Davidson, T.M. Evans, R.N. Slaybaugh, and C.G. Baker.  Massively
  Parallel Solutions to the k-Eigenvalue Problem.
  \textit{Trans. Am. Nucl. Soc.}, {\bf 103}, 2010.
\item T.M. Evans, G.G. Davidson, and R.N. Slaybaugh.  Three-Dimensional Full
  Core Power Calculations for Pressurized Water Reactors.  \textit{Journal of
  Physics: Conference Series, SciDAC 2010}, accepted for publication, 2010.
\item T.M. Evans, K.T. Clarno, and J.E. Morel.  A transport acceleration
  scheme for multigroup discrete ordinates with
  upscattering. \textit{Nuc. Sci. Eng.}, {\bf 165}, 1--13, 2010.
\item T.M. Evans, A.S. Stafford, R.N. Slaybaugh, and K.T. Clarno.  Denovo---A
  new three-dimensional parallel discrete ordinates code in
  SCALE. \textit{Nuc. Tech.}, {\bf 171}, 171--200, 2010.
\item A.M. Ibrahim, D.E. Peplow, T.M. Evans, J.C. Wagner, and
  P.P.H. Wilson. Improving the Mesh Generation Capabilities in the
  SCALE Hybrid Shielding Analysis
  Sequence. \textit{Trans. Am. Nucl. Soc.}, {\bf 100}, 302--303, 2009.
\item T.M. Evans and S.W. Mosher.  A Monte Carlo Synthetic
  Acceleration Method for the Non-Linear, Time-Dependent Diffusion
  Equation.  \textit{International Conference on Mathematics,
    Computational Methods and Reactor Physics}, Saratoga Springs, NY,
  ISBN: 978-0-89448-069-0, American Nuclear Society, LaGrange Park,
  IL, 2009.
\item Ryan G. McClarren, Thomas M. Evans, Robert B. Lowrie, and
  Jefferey D. Densmore.  Semi-Implicit Time Integration for $P_N$
  Thermal Radiative Transfer.  \textit{J. Comp. Phys.}, {\bf 227},
  7561--7586, 2008.
\item J.D. Densmore, T.M. Evans, and M.W. Buksas.  A Hybrid
  Transport-Diffusion Algorithm for Monte Carlo Radiation-Transport
  Simulations on Adaptive-Refinement Meshes in $XY$ Geometry.
  \textit{Nuc. Sci. Eng.}, {\bf 159}, 1--22, 2008.
\item T.M. Evans and J.D. Densmore.  Methods for Coupling Radiation,
  Ion, and Electron Energies In Grey Implicit Monte Carlo.
  \textit{J. Comp. Phys.}, {\bf 225}, 1695--1720, 2007.
\item T.A. Brunner, T.J. Urbatsch, T.M. Evans, and N.A. Gentile.
  Comparison of Four Parallel Algorithms for Domain Decomposed
  Implicit Monte Carlo. \textit{J. Comp. Phys.}, {\bf 212},
  527--539, 2006.
\end{enumerate}

%%---------------------------------------------------------------------------%%

{\bf SYNERGISTIC ACTIVITIES}\\

Evans specializes in the development, implementation, and application of
computational radiation transport as applied in the areas of Nuclear
Engineering, radiation detection, astrophysics, high energy density physics,
and medical applications.  His interests include stochastic and deterministic
transport methods on massively parallel platforms, nonlinear and
time-dependent transport methods, coupled physics including
radiation-hydrodynamics, acceleration and preconditioning techniques,
optimization and performance analysis, and large-scale scientific software
design for parallel codes.  He has published over 50 refereed journal and
conference articles.  He is the primary developer of the Denovo parallel $S_N$
and {\em Shift} Monte Carlo transport codes at Oak Ridge National Laboratory,
which are part of the SCALE package.  He is currently a PI in the CASL
Modeling and Simulation Hub at Oak Ridge.\\

He has served on the technical committee for several Math and Computation
Division (MCD) meetings of the American Nuclear Society and is currently on
the MCD Executive Committee and was MCD Secretary in 2000-2001.\\

%%---------------------------------------------------------------------------%%

{\bf RECENT COLLABORATIONS AND CO-EDITORS}\\

Chris Baker, Oak Ridge National Laboratory;
Roscoe Bartlett, Sandia National Laboratory;
Tom Brunner, Lawerence Livermore National Laboratory;
Kevin Clarno, Oak Ridge National Laboratory;
Greg Davidson, Oak Ridge National Laboratory;
Jeff Densmore, Los Alamos National Laboratory;
Ben Forget, MIT;
Nick Gentile, Lawrence Livermore National Laboratory;
Mike Heroux, Sandia National Laboratory;
Aimee Hungerford, Los Alamos National Laboratory ;
Josh Jarrell, Oak Ridge National Laboratory;
Dana Knoll, Idaho National Laboratory;
Doug Kothe, Oak Ridge National Laboratory;
Ed Larsen, University of Michigan;
Robert Lowrie, Los Alamos National Laboratory;
Ryan McClarren, Texas A\&M University;
Jim Morel, Texas A\&M University;
Scott Mosher, Oak Ridge National Laboratory;
Todd Palmer, Oregan State University;
Roger Pawlowski, Sandia National Laboratory;
Douglas Peplow, Oak Ridge National Laboratory;
Jean Ragusa, Texas A\&M University;
John Turner, Oak Ridge National Laboratory;
Todd Urbatsch, Los Alamos National Laboratory;
John Wagner, Oak Ridge National Laboratory;
Mark Williams, Oak Ridge National Laboratory;
Paul Wilson, University of Wisconsin\\

\underline{PhD Advisees}\\
Rachel Slaybaugh (University of Wisconsin)\\

\underline{Students}:\\
Ahmed Ibrahim (University of Wisconsin);
Peter Maginot (Texas A\&M);
Stuart Slattery (University of Wisconsin)

\end{document}

%%---------------------------------------------------------------------------%%
%% end of CV
%%---------------------------------------------------------------------------%%

